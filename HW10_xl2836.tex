\documentclass[]{article}
\usepackage{lmodern}
\usepackage{amssymb,amsmath}
\usepackage{ifxetex,ifluatex}
\usepackage{fixltx2e} % provides \textsubscript
\ifnum 0\ifxetex 1\fi\ifluatex 1\fi=0 % if pdftex
  \usepackage[T1]{fontenc}
  \usepackage[utf8]{inputenc}
\else % if luatex or xelatex
  \ifxetex
    \usepackage{mathspec}
  \else
    \usepackage{fontspec}
  \fi
  \defaultfontfeatures{Ligatures=TeX,Scale=MatchLowercase}
\fi
% use upquote if available, for straight quotes in verbatim environments
\IfFileExists{upquote.sty}{\usepackage{upquote}}{}
% use microtype if available
\IfFileExists{microtype.sty}{%
\usepackage{microtype}
\UseMicrotypeSet[protrusion]{basicmath} % disable protrusion for tt fonts
}{}
\usepackage[margin=1in]{geometry}
\usepackage{hyperref}
\hypersetup{unicode=true,
            pdftitle={Homework 10},
            pdfauthor={Xinyi Lin},
            pdfborder={0 0 0},
            breaklinks=true}
\urlstyle{same}  % don't use monospace font for urls
\usepackage{color}
\usepackage{fancyvrb}
\newcommand{\VerbBar}{|}
\newcommand{\VERB}{\Verb[commandchars=\\\{\}]}
\DefineVerbatimEnvironment{Highlighting}{Verbatim}{commandchars=\\\{\}}
% Add ',fontsize=\small' for more characters per line
\usepackage{framed}
\definecolor{shadecolor}{RGB}{248,248,248}
\newenvironment{Shaded}{\begin{snugshade}}{\end{snugshade}}
\newcommand{\KeywordTok}[1]{\textcolor[rgb]{0.13,0.29,0.53}{\textbf{#1}}}
\newcommand{\DataTypeTok}[1]{\textcolor[rgb]{0.13,0.29,0.53}{#1}}
\newcommand{\DecValTok}[1]{\textcolor[rgb]{0.00,0.00,0.81}{#1}}
\newcommand{\BaseNTok}[1]{\textcolor[rgb]{0.00,0.00,0.81}{#1}}
\newcommand{\FloatTok}[1]{\textcolor[rgb]{0.00,0.00,0.81}{#1}}
\newcommand{\ConstantTok}[1]{\textcolor[rgb]{0.00,0.00,0.00}{#1}}
\newcommand{\CharTok}[1]{\textcolor[rgb]{0.31,0.60,0.02}{#1}}
\newcommand{\SpecialCharTok}[1]{\textcolor[rgb]{0.00,0.00,0.00}{#1}}
\newcommand{\StringTok}[1]{\textcolor[rgb]{0.31,0.60,0.02}{#1}}
\newcommand{\VerbatimStringTok}[1]{\textcolor[rgb]{0.31,0.60,0.02}{#1}}
\newcommand{\SpecialStringTok}[1]{\textcolor[rgb]{0.31,0.60,0.02}{#1}}
\newcommand{\ImportTok}[1]{#1}
\newcommand{\CommentTok}[1]{\textcolor[rgb]{0.56,0.35,0.01}{\textit{#1}}}
\newcommand{\DocumentationTok}[1]{\textcolor[rgb]{0.56,0.35,0.01}{\textbf{\textit{#1}}}}
\newcommand{\AnnotationTok}[1]{\textcolor[rgb]{0.56,0.35,0.01}{\textbf{\textit{#1}}}}
\newcommand{\CommentVarTok}[1]{\textcolor[rgb]{0.56,0.35,0.01}{\textbf{\textit{#1}}}}
\newcommand{\OtherTok}[1]{\textcolor[rgb]{0.56,0.35,0.01}{#1}}
\newcommand{\FunctionTok}[1]{\textcolor[rgb]{0.00,0.00,0.00}{#1}}
\newcommand{\VariableTok}[1]{\textcolor[rgb]{0.00,0.00,0.00}{#1}}
\newcommand{\ControlFlowTok}[1]{\textcolor[rgb]{0.13,0.29,0.53}{\textbf{#1}}}
\newcommand{\OperatorTok}[1]{\textcolor[rgb]{0.81,0.36,0.00}{\textbf{#1}}}
\newcommand{\BuiltInTok}[1]{#1}
\newcommand{\ExtensionTok}[1]{#1}
\newcommand{\PreprocessorTok}[1]{\textcolor[rgb]{0.56,0.35,0.01}{\textit{#1}}}
\newcommand{\AttributeTok}[1]{\textcolor[rgb]{0.77,0.63,0.00}{#1}}
\newcommand{\RegionMarkerTok}[1]{#1}
\newcommand{\InformationTok}[1]{\textcolor[rgb]{0.56,0.35,0.01}{\textbf{\textit{#1}}}}
\newcommand{\WarningTok}[1]{\textcolor[rgb]{0.56,0.35,0.01}{\textbf{\textit{#1}}}}
\newcommand{\AlertTok}[1]{\textcolor[rgb]{0.94,0.16,0.16}{#1}}
\newcommand{\ErrorTok}[1]{\textcolor[rgb]{0.64,0.00,0.00}{\textbf{#1}}}
\newcommand{\NormalTok}[1]{#1}
\usepackage{graphicx,grffile}
\makeatletter
\def\maxwidth{\ifdim\Gin@nat@width>\linewidth\linewidth\else\Gin@nat@width\fi}
\def\maxheight{\ifdim\Gin@nat@height>\textheight\textheight\else\Gin@nat@height\fi}
\makeatother
% Scale images if necessary, so that they will not overflow the page
% margins by default, and it is still possible to overwrite the defaults
% using explicit options in \includegraphics[width, height, ...]{}
\setkeys{Gin}{width=\maxwidth,height=\maxheight,keepaspectratio}
\IfFileExists{parskip.sty}{%
\usepackage{parskip}
}{% else
\setlength{\parindent}{0pt}
\setlength{\parskip}{6pt plus 2pt minus 1pt}
}
\setlength{\emergencystretch}{3em}  % prevent overfull lines
\providecommand{\tightlist}{%
  \setlength{\itemsep}{0pt}\setlength{\parskip}{0pt}}
\setcounter{secnumdepth}{0}
% Redefines (sub)paragraphs to behave more like sections
\ifx\paragraph\undefined\else
\let\oldparagraph\paragraph
\renewcommand{\paragraph}[1]{\oldparagraph{#1}\mbox{}}
\fi
\ifx\subparagraph\undefined\else
\let\oldsubparagraph\subparagraph
\renewcommand{\subparagraph}[1]{\oldsubparagraph{#1}\mbox{}}
\fi

%%% Use protect on footnotes to avoid problems with footnotes in titles
\let\rmarkdownfootnote\footnote%
\def\footnote{\protect\rmarkdownfootnote}

%%% Change title format to be more compact
\usepackage{titling}

% Create subtitle command for use in maketitle
\newcommand{\subtitle}[1]{
  \posttitle{
    \begin{center}\large#1\end{center}
    }
}

\setlength{\droptitle}{-2em}

  \title{Homework 10}
    \pretitle{\vspace{\droptitle}\centering\huge}
  \posttitle{\par}
    \author{Xinyi Lin}
    \preauthor{\centering\large\emph}
  \postauthor{\par}
      \predate{\centering\large\emph}
  \postdate{\par}
    \date{5/5/2019}


\begin{document}
\maketitle

\begin{Shaded}
\begin{Highlighting}[]
\KeywordTok{library}\NormalTok{(survival)}
\KeywordTok{library}\NormalTok{(MASS)}
\KeywordTok{library}\NormalTok{(survminer)}
\KeywordTok{library}\NormalTok{(KMsurv)}
\KeywordTok{library}\NormalTok{(tidyverse)}
\end{Highlighting}
\end{Shaded}

\subsection{Problem 1}\label{problem-1}

Input data

\begin{Shaded}
\begin{Highlighting}[]
\NormalTok{time =}\StringTok{ }\KeywordTok{c}\NormalTok{(}\DecValTok{4}\NormalTok{, }\DecValTok{12}\NormalTok{, }\DecValTok{15}\NormalTok{, }\DecValTok{21}\NormalTok{, }\DecValTok{23}\NormalTok{, }\DecValTok{2}\NormalTok{, }\DecValTok{6}\NormalTok{, }\DecValTok{8}\NormalTok{, }\DecValTok{10}\NormalTok{, }\DecValTok{19}\NormalTok{)}
\NormalTok{cens =}\StringTok{ }\KeywordTok{c}\NormalTok{(}\DecValTok{1}\NormalTok{, }\DecValTok{0}\NormalTok{, }\DecValTok{1}\NormalTok{, }\DecValTok{0}\NormalTok{, }\DecValTok{1}\NormalTok{, }\DecValTok{1}\NormalTok{, }\DecValTok{0}\NormalTok{, }\DecValTok{0}\NormalTok{, }\DecValTok{1}\NormalTok{, }\DecValTok{1}\NormalTok{)}
\NormalTok{group =}\StringTok{ }\KeywordTok{c}\NormalTok{(}\KeywordTok{rep}\NormalTok{(}\DecValTok{1}\NormalTok{, }\DecValTok{5}\NormalTok{), }\KeywordTok{rep}\NormalTok{(}\DecValTok{2}\NormalTok{, }\DecValTok{5}\NormalTok{))}
\NormalTok{data1 =}\StringTok{ }\KeywordTok{data.frame}\NormalTok{(}\DataTypeTok{time =}\NormalTok{ time, }\DataTypeTok{cens =}\NormalTok{ cens, }\DataTypeTok{group =}\NormalTok{ group)}
\NormalTok{data1}
\end{Highlighting}
\end{Shaded}

\begin{verbatim}
##    time cens group
## 1     4    1     1
## 2    12    0     1
## 3    15    1     1
## 4    21    0     1
## 5    23    1     1
## 6     2    1     2
## 7     6    0     2
## 8     8    0     2
## 9    10    1     2
## 10   19    1     2
\end{verbatim}

Using the log-rank test to test hypotheses.

\(H_0: h_1(t) = h_2(t)\) for all t; \(H_1: h_1(t) \neq h_2(t)\) for some
t.

\begin{Shaded}
\begin{Highlighting}[]
\KeywordTok{survdiff}\NormalTok{(}\KeywordTok{Surv}\NormalTok{(time,cens)}\OperatorTok{~}\NormalTok{group, }\DataTypeTok{data=}\NormalTok{data1) }\CommentTok{# log rank test}
\end{Highlighting}
\end{Shaded}

\begin{verbatim}
## Call:
## survdiff(formula = Surv(time, cens) ~ group, data = data1)
## 
##         N Observed Expected (O-E)^2/E (O-E)^2/V
## group=1 5        3     4.14     0.313      1.15
## group=2 5        3     1.86     0.697      1.15
## 
##  Chisq= 1.1  on 1 degrees of freedom, p= 0.3
\end{verbatim}

\begin{Shaded}
\begin{Highlighting}[]
\CommentTok{#plot(survfit(Surv(time,cens)~group, data = data)) }
\KeywordTok{ggsurvplot}\NormalTok{( }\KeywordTok{survfit}\NormalTok{(}\KeywordTok{Surv}\NormalTok{(time, cens) }\OperatorTok{~}\StringTok{ }\NormalTok{group, }\DataTypeTok{data =}\NormalTok{ data1), }\DataTypeTok{conf.int=}\OtherTok{TRUE}\NormalTok{)}
\end{Highlighting}
\end{Shaded}

\includegraphics{HW10_xl2836_files/figure-latex/unnamed-chunk-3-1.pdf}

Calculating by using R, we can get that \(Z^2 = 1.1\) and corresponding
\(p-value = 0.3\). As p-value is larger than 0.05, we fail to reject the
null hypothesis and conclude that there \(h_1(t) = h_2(t)\) for all t.
The plot of survival probability of two group are shown above.

\subsection{Problem 2}\label{problem-2}

\begin{Shaded}
\begin{Highlighting}[]
\KeywordTok{data}\NormalTok{(}\StringTok{"kidtran"}\NormalTok{)}
\KeywordTok{head}\NormalTok{(kidtran)}
\end{Highlighting}
\end{Shaded}

\begin{verbatim}
##   obs time delta gender race age
## 1   1    1     0      1    1  46
## 2   2    5     0      1    1  51
## 3   3    7     1      1    1  55
## 4   4    9     0      1    1  57
## 5   5   13     0      1    1  45
## 6   6   13     0      1    1  43
\end{verbatim}

\begin{Shaded}
\begin{Highlighting}[]
\NormalTok{female_kid =}\StringTok{ }\NormalTok{kidtran }\OperatorTok\StringTok{ }
\StringTok{  }\KeywordTok{filter}\NormalTok{(gender }\OperatorTok{==}\StringTok{ }\DecValTok{2}\NormalTok{)}
\NormalTok{male_kid =}\StringTok{ }\NormalTok{kidtran }\OperatorTok\StringTok{ }
\StringTok{  }\KeywordTok{filter}\NormalTok{(gender }\OperatorTok{==}\StringTok{ }\DecValTok{1}\NormalTok{)}
\end{Highlighting}
\end{Shaded}

For female:

\begin{Shaded}
\begin{Highlighting}[]
\KeywordTok{ggsurvplot}\NormalTok{( }\KeywordTok{survfit}\NormalTok{(}\KeywordTok{Surv}\NormalTok{(time, delta) }\OperatorTok{~}\StringTok{ }\NormalTok{race, }\DataTypeTok{data =}\NormalTok{ female_kid), }\DataTypeTok{conf.int=}\OtherTok{TRUE}\NormalTok{)}
\end{Highlighting}
\end{Shaded}

\includegraphics{HW10_xl2836_files/figure-latex/unnamed-chunk-5-1.pdf}

\begin{Shaded}
\begin{Highlighting}[]
\KeywordTok{ggsurvplot}\NormalTok{( }\KeywordTok{survfit}\NormalTok{(}\KeywordTok{Surv}\NormalTok{(time, delta) }\OperatorTok{~}\StringTok{ }\NormalTok{race, }\DataTypeTok{data =}\NormalTok{ male_kid), }\DataTypeTok{conf.int=}\OtherTok{TRUE}\NormalTok{)}
\end{Highlighting}
\end{Shaded}

\includegraphics{HW10_xl2836_files/figure-latex/unnamed-chunk-6-1.pdf}

\subsection{Problem 3}\label{problem-3}

Get data.

\begin{Shaded}
\begin{Highlighting}[]
\KeywordTok{data}\NormalTok{(}\StringTok{"larynx"}\NormalTok{)}
\NormalTok{larynx_data =}\StringTok{ }\NormalTok{larynx }\OperatorTok\StringTok{ }
\StringTok{  }\KeywordTok{mutate}\NormalTok{(}\DataTypeTok{z1 =} \KeywordTok{ifelse}\NormalTok{(stage }\OperatorTok{==}\StringTok{ }\DecValTok{2}\NormalTok{, }\DecValTok{1}\NormalTok{, }\DecValTok{0}\NormalTok{),}
         \DataTypeTok{z2 =} \KeywordTok{ifelse}\NormalTok{(stage }\OperatorTok{==}\StringTok{ }\DecValTok{3}\NormalTok{, }\DecValTok{1}\NormalTok{, }\DecValTok{0}\NormalTok{), }
         \DataTypeTok{z3 =} \KeywordTok{ifelse}\NormalTok{(stage }\OperatorTok{==}\StringTok{ }\DecValTok{4}\NormalTok{, }\DecValTok{1}\NormalTok{, }\DecValTok{0}\NormalTok{))}
\KeywordTok{head}\NormalTok{(larynx_data)}
\end{Highlighting}
\end{Shaded}

\begin{verbatim}
##   stage time age diagyr delta z1 z2 z3
## 1     1  0.6  77     76     1  0  0  0
## 2     1  1.3  53     71     1  0  0  0
## 3     1  2.4  45     71     1  0  0  0
## 4     1  2.5  57     78     0  0  0  0
## 5     1  3.2  58     74     1  0  0  0
## 6     1  3.2  51     77     0  0  0  0
\end{verbatim}

We fit following model(using Breslow method for tie handling):
\[h(t) = h_0(t)exp(\beta_1Z_1 + \beta_2Z_2 + \beta_3Z_3 + \beta_4Z_4 + \beta_5Z_1\times Z_4)\]

\begin{Shaded}
\begin{Highlighting}[]
\NormalTok{fit=}\KeywordTok{coxph}\NormalTok{(}\KeywordTok{Surv}\NormalTok{(time,delta)}\OperatorTok{~}\NormalTok{z1}\OperatorTok{+}\NormalTok{z2}\OperatorTok{+}\NormalTok{z3}\OperatorTok{+}\NormalTok{age}\OperatorTok{+}\NormalTok{z1}\OperatorTok{*}\NormalTok{age,}\DataTypeTok{data=}\NormalTok{larynx_data,}\DataTypeTok{ties=}\StringTok{'breslow'}\NormalTok{)}
\KeywordTok{summary}\NormalTok{(fit)}
\end{Highlighting}
\end{Shaded}

\begin{verbatim}
## Call:
## coxph(formula = Surv(time, delta) ~ z1 + z2 + z3 + age + z1 * 
##     age, data = larynx_data, ties = "breslow")
## 
##   n= 90, number of events= 50 
## 
##              coef  exp(coef)   se(coef)      z Pr(>|z|)    
## z1     -7.3820143  0.0006223  3.4027542 -2.169   0.0301 *  
## z2      0.6218044  1.8622853  0.3558078  1.748   0.0805 .  
## z3      1.7534270  5.7743576  0.4239595  4.136 3.54e-05 ***
## age     0.0059729  1.0059908  0.0148792  0.401   0.6881    
## z1:age  0.1116674  1.1181409  0.0476728  2.342   0.0192 *  
## ---
## Signif. codes:  0 '***' 0.001 '**' 0.01 '*' 0.05 '.' 0.1 ' ' 1
## 
##        exp(coef) exp(-coef) lower .95 upper .95
## z1     0.0006223  1606.8231 7.900e-07    0.4903
## z2     1.8622853     0.5370 9.272e-01    3.7403
## z3     5.7743576     0.1732 2.516e+00   13.2550
## age    1.0059908     0.9940 9.771e-01    1.0358
## z1:age 1.1181409     0.8943 1.018e+00    1.2277
## 
## Concordance= 0.682  (se = 0.045 )
## Rsquare= 0.235   (max possible= 0.988 )
## Likelihood ratio test= 24.11  on 5 df,   p=2e-04
## Wald test            = 23.77  on 5 df,   p=2e-04
## Score (logrank) test = 27.98  on 5 df,   p=4e-05
\end{verbatim}

\begin{description}
\item[Explain of results:]
the log hazard ratio for subjects in stage II versus stage I is
(-7.382+0.112*a) given they have same age, where a is the age of
subjects.

the log hazard ratio for subjects in stage III versus stage I is 0.621
given they have same age.

the log hazard ratio for subjects in stage IV versus stage I is 1.753
given they have same age.

when patients are not in stage II,the log hazard ratio for subjects
with one unit changes in age is 0.621 given they are in same stage.

when patients are in stage II,the log hazard ratio for subjects with
one unit changes in age is 0.118 given they are in same stage.
\end{description}

Relative risk:

For the hazard of dying for a stage II patient of age 50 is
\[h_2(t) = h_0(t)exp(-7.382\times1 + 0.111\times1\times50)\]

For the hazard of dying for a stage I patient of age 50 is
\[h_1(t) = h_0(t)exp(0)\]

So the hazard ratio is \[HR(t) = \frac{h_2(t)}{h_1(t)} = 0.16\]


\end{document}
